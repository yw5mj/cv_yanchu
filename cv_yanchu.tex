%%%%%%%%%%%%%%%%%%%%%%%%%%%%%%%%%%%%%%%%%
% Wilson Resume/CV
% XeLaTeX Template
% Version 1.0 (22/1/2015)
%
% This template has been downloaded from:
% http://www.LaTeXTemplates.com
%
% Original author:
% Howard Wilson (https://github.com/watsonbox/cv_template_2004) with
% extensive modifications by Vel (vel@latextemplates.com)
%
% License:
% CC BY-NC-SA 3.0 (http://creativecommons.org/licenses/by-nc-sa/3.0/)
%
%%%%%%%%%%%%%%%%%%%%%%%%%%%%%%%%%%%%%%%%%

%----------------------------------------------------------------------------------------
%	PACKAGES AND OTHER DOCUMENT CONFIGURATIONS
%----------------------------------------------------------------------------------------

\documentclass[10pt]{article} % Default font size

%%%%%%%%%%%%%%%%%%%%%%%%%%%%%%%%%%%%%%%%%
% Wilson Resume/CV
% Structure Specification File
% Version 1.0 (22/1/2015)
%
% This file has been downloaded from:
% http://www.LaTeXTemplates.com
%
% License:
% CC BY-NC-SA 3.0 (http://creativecommons.org/licenses/by-nc-sa/3.0/)
%
%%%%%%%%%%%%%%%%%%%%%%%%%%%%%%%%%%%%%%%%%

%----------------------------------------------------------------------------------------
%	PACKAGES AND OTHER DOCUMENT CONFIGURATIONS
%----------------------------------------------------------------------------------------

\usepackage[a4paper, hmargin=25mm, vmargin=30mm, top=20mm]{geometry} % Use A4 paper and set margins

\usepackage{fancyhdr} % Customize the header and footer

\usepackage{lastpage} % Required for calculating the number of pages in the document

\usepackage[colorlinks=false,pdfborder=000]{hyperref} % Colors for links, text and headings

\setcounter{secnumdepth}{0} % Suppress section numbering

%\usepackage[proportional,scaled=1.064]{erewhon} % Use the Erewhon font
%\usepackage[erewhon,vvarbb,bigdelims]{newtxmath} % Use the Erewhon font
\usepackage[utf8]{inputenc} % Required for inputting international characters
\usepackage[T1]{fontenc} % Output font encoding for international characters

\usepackage{color} % Required for custom colors
\definecolor{slateblue}{rgb}{0.17,0.22,0.34}

\usepackage{sectsty} % Allows customization of titles
\sectionfont{\color{slateblue}} % Color section titles

\fancypagestyle{plain}{\fancyhf{}\cfoot{\thepage\ of \pageref{LastPage}}} % Define a custom page style
\pagestyle{plain} % Use the custom page style through the document
\renewcommand{\headrulewidth}{0pt} % Disable the default header rule
\renewcommand{\footrulewidth}{0pt} % Disable the default footer rule

\setlength\parindent{0pt} % Stop paragraph indentation

% Non-indenting itemize
\newenvironment{itemize-noindent}
{\setlength{\leftmargini}{0em}\begin{itemize}}
{\end{itemize}}

% Text width for tabbing environments
\newlength{\smallertextwidth}
\setlength{\smallertextwidth}{\textwidth}
\addtolength{\smallertextwidth}{-2cm}

\newcommand{\sqbullet}{~\vrule height 1ex width .8ex depth -.2ex} % Custom square bullet point definition

%----------------------------------------------------------------------------------------
%	MAIN HEADER COMMAND
%----------------------------------------------------------------------------------------

\renewcommand{\title}[1]{
{\huge{\color{slateblue}\textbf{#1}}}\\ % Header section name and color
\rule{\textwidth}{0.5mm}\\ % Rule under the header
}

%----------------------------------------------------------------------------------------
%	JOB COMMAND
%----------------------------------------------------------------------------------------

\newcommand{\job}[6]{
\begin{tabbing}
\hspace{2cm} \= \kill
\textbf{#1} \> \href{#4}{#3} \\
\textbf{#2} \>\+ \textit{#5} \\
\begin{minipage}{\smallertextwidth}
\vspace{2mm}
#6
\end{minipage}
\end{tabbing}
\vspace{2mm}
}

%----------------------------------------------------------------------------------------
%	SKILL GROUP COMMAND
%----------------------------------------------------------------------------------------

\newcommand{\skillgroup}[2]{
\begin{tabbing}
\hspace{5mm} \= \kill
\sqbullet \>\+ \textbf{#1} \\
\begin{minipage}{\smallertextwidth}
\vspace{2mm}
#2
\end{minipage}
\end{tabbing}
}

%----------------------------------------------------------------------------------------
%	INTERESTS GROUP COMMAND
%-----------------------------------------------------------------------------------------

\newcommand{\interestsgroup}[1]{
\begin{tabbing}
\hspace{5mm} \= \kill
#1
\end{tabbing}
\vspace{-10mm}
}

\newcommand{\interest}[1]{\sqbullet \> \textbf{#1}\\[3pt]} % Define a custom command for individual interests

%----------------------------------------------------------------------------------------
%	TABBED BLOCK COMMAND
%----------------------------------------------------------------------------------------

\newcommand{\tabbedblock}[1]{
\begin{tabbing}
\hspace{2cm} \= \hspace{4cm} \= \kill
#1
\end{tabbing}
}
 % Include the file specifying document layout
%----------------------------------------------------------------------------------------

\begin{document}

%----------------------------------------------------------------------------------------
%	NAME AND CONTACT INFORMATION
%----------------------------------------------------------------------------------------

\title{Yanchu Wang\, \small{Résumé}} % Print the main header

%------------------------------------------------

\parbox{0.5\textwidth}{ % First block
\begin{tabbing} % Enables tabbing
\hspace{3cm} \= \hspace{4cm} \= \kill % Spacing within the block
{\bf Address} \> Department of Physics \\
\> University of Virginia\\ % Address line 1
\> Charlottesville, VA, 22904 \\ % Address line 2

\end{tabbing}}
\hfill % Horizontal space between the two blocks
\parbox{0.5\textwidth}{ % Second block
\begin{tabbing} % Enables tabbing
\hspace{3cm} \= \hspace{4cm} \= \kill % Spacing within the block
{\bf Cellphone} \> (434)448-2055 \\ % Mobile phone
{\bf Email} \> \href{mailto:yw5mj@virginia.edu}{\textit{yw5mj@virginia.edu}} \\ % Email address
{\bf Linkedin} \> \href{https://www.linkedin.com/in/yanchu-wang-46040289/}{\textit{click me}} \\ % Email address
\end{tabbing}}

%----------------------------------------------------------------------------------------
%	PERSONAL PROFILE
%----------------------------------------------------------------------------------------

\section{Overview}

I am a graduate student at physics department of University of Virginia (UVA), working on physics data analysis for the Compact Muon Solenoid (CMS) Experiment, built on the Large Hadron Collider (LHC) located at CERN, Switzerland. I expect to graduate in late 2018 and now I am looking for Quantitative/data analysis related working opportunities in the USA.

%----------------------------------------------------------------------------------------
%	EDUCATION SECTION
%----------------------------------------------------------------------------------------

\section{Education \& Background}

\tabbedblock{
\bf{2013-2018} \> Ph.D candidate in Physics \\
\>University of Virginia (UVA), Charlottesville, VA\\[5pt]
\>\textit{Advisor} \> Prof. Bob Hirosky (UVA)\\
\>\textit{Thesis (In progress)} \> Search for diboson resonances in the $2l2\nu$ final state \\
\>\textit{GPA} \>  3.76/4.00\\
}

%------------------------------------------------
\tabbedblock{
\bf{2015-2017} \> Visiting Scholar\\
\>CERN, Geneva, Switzerland\\
}
%------------------------------------------------
\tabbedblock{
\bf{2009-2013} \> B.S. in Physics\\
\>University of Science and Technology of China (USTC), Hefei, China\\[5pt]
\>\textit{GPA} \>  3.71/4.00\\
}

%----------------------------------------------------------------------------------------
%	EMPLOYMENT HISTORY SECTION
%----------------------------------------------------------------------------------------

\section{Research Experiences}
\job
{Dec 2015 -}{Present}
{Working on CMS data analysis:}
{Search for diboson resonances in the $2l2\nu$ final state
\begin{itemize}
\item PhD thesis topic
\item Paper has been published on JHEP (\href{https://arxiv.org/abs/1711.04370}{\textit{arXiv link}})
\end{itemize}}
%------------------------------------------------
\job
{Jan 2017 -}{Mar 2018}
{Working as Monte Carlo Simulation Contact Person in CMS Physics Analysis Group}
{
\begin{itemize}
\item Responsible for generating MC data samples and reviewing MC simulation usage for multiple analyses aiming at Beyond Standard Model particle searches
\item Key position for CMS physics analyses
\end{itemize}}
%------------------------------------------------
\job
{Feb 2016 -}{Jun 2017}
{Worked as Detector On Call Expert for CMS Hadron Calibrator (HCAL)}
{
\begin{itemize}
\item Responsible for solving urgent detector issues, coordinating HCAL activities and plans
\item Demonstrated efficient problem-solving and communication abilities
\end{itemize}}
%------------------------------------------------
\job
{Jun 2015 -}{Feb 2016}
{Worked on CMS HCAL frontend electrics upgrading tests}
{Developed a python-based software system for over 150 electronic chips to
\begin{itemize}
\item monitor the chip health status and update summary plots automatically online
\item test various properties of the chips and synchronize the test results into an SQL database
\end{itemize}
}

\section{Conference Talks}
\job
{May 2018}{}
{Searches for heavy resonances decaying into Z, W and Higgs bosons at CMS}
{\begin{itemize}
\item Phenomenology Symposium (PHENO) @ Pittsburgh, PA
\item \href{https://indico.cern.ch/event/699148/contributions/2986197/}{\textit{Contribution link}}
\end{itemize}
}
\job
{Aug 2017}{}
{Searches for new resonances decaying into diboson final states with large missing transverse momentum}
{\begin{itemize}
\item APS DPF Meeting @ Fermilab, IL
\item \href{https://indico.fnal.gov/event/11999/session/10/contribution/56}{\textit{Contribution link}}
\end{itemize}
}

\section{Academic Honors}
\begin{tabbing} % Enables tabbing
\hspace{2cm} \= \kill
\textbf{May 2014} \> Distinction in the UVA PhD qualification (top 2/15) \\
\textbf{Sep 2013} \> UVA Physics Department Fellowship \\
%\textbf{Jun 2012} \> Scholarship for outstanding academic performance (USTC) \\
%\textbf{Jun 2011} \> Scholarship for outstanding academic performance (USTC) \\
%\textbf{Jun 2010} \> Scholarship for outstanding academic performance (USTC) \\
\textbf{2009\,-2013} \> Scholarship for outstanding academic performance (USTC) every academic year\\
\end{tabbing}
%----------------------------------------------------------------------------------------
%	IT/COMPUTING SKILLS SECTION
%----------------------------------------------------------------------------------------

\section{Programming Skills}
\textbf{python}
\small{
My primary programming tool. It is the main scripting language I used for my physics analysis and detector hardware tests.
}\\


\textbf{Bash/Linux}
\small{
Linux is my primary operating system. I am familiar with Linux structure and bash scripting.
}\\


\textbf{(Py)ROOT@CERN}
\small{
ROOT is a scientific software framework designed by CERN for particle physics data analysis. I used ROOT and PyROOT (ROOT as a python module) in almost every research project I was involved in.
}\\


\textbf{C++}
\small{
The most common programming language in CMS. I started learning C++ in college and use it all the time.
}

\end{document}
