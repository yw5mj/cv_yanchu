%%%%%%%%%%%%%%%%%%%%%%%%%%%%%%%%%%%%%%%%%
% XeLaTeX Template
% Version 1.0 (22/1/2015)
%%%%%%%%%%%%%%%%%%%%%%%%%%%%%%%%%%%%%%%%%

%----------------------------------------------------------------------------------------
%	PACKAGES AND OTHER DOCUMENT CONFIGURATIONS
%----------------------------------------------------------------------------------------

\documentclass[10pt]{article} % Default font size

\input{structure.tex} % Include the file specifying document layout
%----------------------------------------------------------------------------------------

\begin{document}

%----------------------------------------------------------------------------------------
%	NAME AND CONTACT INFORMATION
%----------------------------------------------------------------------------------------

\title{Yanchu Wang\, \small{Résumé}} % Print the main header

%------------------------------------------------

\parbox{0.5\textwidth}{ % First block
\begin{tabbing} % Enables tabbing
\hspace{3cm} \= \hspace{4cm} \= \kill % Spacing within the block
{\bf Address} \> Department of Physics \\
\> University of Virginia\\ % Address line 1
\> Charlottesville, VA, 22904 \\ % Address line 2

\end{tabbing}}
\hfill % Horizontal space between the two blocks
\parbox{0.5\textwidth}{ % Second block
\begin{tabbing} % Enables tabbing
\hspace{3cm} \= \hspace{4cm} \= \kill % Spacing within the block
{\bf Cellphone} \> (434)448-2055 \\ % Mobile phone
{\bf Email} \> \href{mailto:yw5mj@virginia.edu}{\textit{yw5mj@virginia.edu}} \\ % Email address
{\bf Linkedin} \> \href{https://www.linkedin.com/in/yanchu-wang-46040289/}{\textit{click me}} \\ % Email address
\end{tabbing}}

%----------------------------------------------------------------------------------------
%	PERSONAL PROFILE
%----------------------------------------------------------------------------------------

\section{Overview}

I am a graduate student at the University of Virginia (UVA) majoring in physics. I have been working on physics data analysis for the Compact Muon Solenoid (CMS) Experiment, built on the Large Hadron Collider (LHC) located at CERN, Switzerland. I expect to graduate in December of 2018 and now I am looking for data analysis related working opportunities in the USA.

%----------------------------------------------------------------------------------------
%	EDUCATION SECTION
%----------------------------------------------------------------------------------------

\section{Education \& Background}

\tabbedblock{
\bf{2013-2018} \> Ph.D candidate in Physics \\
\>University of Virginia (UVA), Charlottesville, VA\\[5pt]
\>\textit{Advisor} \> Prof. Bob Hirosky (UVA)\\
\>\textit{Thesis topic} \> Search for diboson resonances in the $2l2\nu$ final state \\
\>\textit{GPA} \>  3.76\\
}

%------------------------------------------------
\tabbedblock{
\bf{2015-2017} \> Visiting Scholar\\
\>CERN, Geneva, Switzerland\\
}
%------------------------------------------------
\tabbedblock{
\bf{2009-2013} \> B.S. in Physics\\
\>University of Science and Technology of China (USTC), Hefei, China\\[5pt]
\>\textit{GPA} \>  3.71\\
}


%----------------------------------------------------------------------------------------
%	IT/COMPUTING SKILLS SECTION
%----------------------------------------------------------------------------------------
\section{Programming Skills}
\textbf{python}
\small{
My primary programming tool. It is the main scripting language I used for my physics analysis and detector hardware tests.
}\\

\textbf{Bash/Linux}
\small{
Linux is my primary operating system. I am familiar with Linux structure and bash scripting.
}\\

\textbf{(Py)ROOT@CERN}
\small{
ROOT is a scientific software framework designed by CERN for particle physics data analysis. I used ROOT and PyROOT (ROOT as a python module) in almost every research project I was involved in.
}\\

\textbf{C++}
\small{
The most common programming language in CMS. I started learning C++ in college and use it all the time.
}\\

\textbf{SQL}
\small{
I learned SQL in college and used it for the data management of some detector hardware projects. 
}

%\section{Programming Skills}
%\textbf{Proficient:} python, bash, ROOT@CERN \\

%\textbf{Working knowledge:} C++, SQL \\

%----------------------------------------------------------------------------------------
%	EMPLOYMENT HISTORY SECTION
%----------------------------------------------------------------------------------------

\section{Research Experiences}
\job
{Dec 2015 -}{Present}
{\textbf{Working on CMS data analyses}}
{
The goal is to search new particles (signal) from the known physics processes (background). Each dataset we are working on contains $\sim$100 million proton collision events with the size of $\sim$1TB. After data cleaning and skimming, Monte Carlo simulation and data-driven methods are used for background modeling, and classification methods are used to distinguish signal and background. The analysis software frameworks are written mainly in python.
%\item My thesis analysis has been published on JHEP (\href{https://arxiv.org/abs/1711.04370}{\textit{arXiv link}}).
}
%------------------------------------------------
\job
{Jan 2017 -}{Mar 2018}
{\textbf{Working as Monte Carlo Simulation Contact Person in CMS Physics Analysis Group}}
{
\begin{itemize}
\item Responsible for generating MC simulation samples and reviewing the usage of MC datasets as well as corresponding uncertainties for multiple analyses aiming at Beyond Standard Model particle searches
\item Key role for CMS physics analyses
\end{itemize}}
%------------------------------------------------
\job
{Feb 2016 -}{Jun 2017}
{\textbf{Worked as on-call Detector Expert for CMS Hadron Calorimeter (HCAL)}}
{
\begin{itemize}
\item Responsible for solving urgent detector issues 24/7, coordinating HCAL activities and plans
\item Demonstrated efficient problem-solving and communication abilities
\end{itemize}}
%------------------------------------------------
\job
{Jun 2015 -}{Feb 2016}
{\textbf{Worked on CMS HCAL frontend electronics upgrading tests}}
{
Developed python-based software for over 150 electronic chips to:
\begin{itemize}
\item monitor the chip health status and update summary plots automatically online
\item test various properties of the chips and synchronize the test results with an SQL database
\end{itemize}
}


\section{Academic Honors}
\begin{tabbing} % Enables tabbing
\hspace{2cm} \= \kill
\textbf{May 2014} \> Distinction in the UVA PhD qualification (top 2/15) \\
\textbf{Sep 2013} \> UVA Physics Department Fellowship \\
%\textbf{Jun 2012} \> Scholarship for outstanding academic performance (USTC) \\
%\textbf{Jun 2011} \> Scholarship for outstanding academic performance (USTC) \\
%\textbf{Jun 2010} \> Scholarship for outstanding academic performance (USTC) \\
\textbf{2009\,-2013} \> Scholarship for outstanding academic performance (USTC) every academic year \\
\end{tabbing}

\section{Conference Talks}
\job
{May 2018}{}
{Searches for heavy resonances decaying into Z, W and Higgs bosons at CMS}
{\begin{itemize}
\item Phenomenology Symposium (PHENO) @ Pittsburgh, PA
\item \href{https://indico.cern.ch/event/699148/contributions/2986197/}{\textit{Contribution link}}
\end{itemize}
}
\job
{Aug 2017}{}
{Searches for new resonances decaying into diboson final states with large missing transverse momentum}
{\begin{itemize}
\item APS DPF Meeting @ Fermilab, IL
\item \href{https://indico.fnal.gov/event/11999/session/10/contribution/56}{\textit{Contribution link}}
\end{itemize}
}

\section{Selected Publications}
\begin{itemize}
\item \textit{"Search for ZZ resonances in the 2$\ell$2$\nu$ final state in proton-proton collisions at 13 TeV"}, CMS Collaboration, JHEP03(2018)003

\item \textit{"Search for natural supersymmetry in events with top quark pairs and photons in pp collisions at $\sqrt{s}=8$ TeV"}, CMS Collaboration, JHEP03(2018)167

\item \textit{"Measurements of properties of the Higgs boson decaying into the four-lepton final state in pp collisions at $\sqrt{s}=13$ TeV"}, CMS Collaboration, JHEP11(2017)047
\end{itemize}

\end{document}
